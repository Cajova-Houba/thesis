%%%%%%%%%%%%%%%%%%%%%%%%%%%%%%%%%%%%%%%%%%%%%%%%%%%%%%%%%%
%
% Vzor pro sazbu kvalifikační práce
%
% Západočeská univerzita v Plzni
% Fakulta aplikovaných věd
% Katedra informatiky a výpočetní techniky
%
% Petr Lobaz, lobaz@kiv.zcu.cz, 2016/03/14
%
%%%%%%%%%%%%%%%%%%%%%%%%%%%%%%%%%%%%%%%%%%%%%%%%%%%%%%%%%%

% Možné jazyky práce: czech, english
% Možné typy práce: BP (bakalářská), DP (diplomová)
\documentclass[czech,DP]{thesiskiv}

% Definujte údaje pro vstupní strany
%
% Jméno a příjmení; kvůli textu prohlášení určete, 
% zda jde o mužské, nebo ženské jméno.
\author{Zdeněk Valeš}
\declarationmale

%alternativa: 
%\declarationfemale

% Název práce
\title{Určování nahraditelnosti a\\kompatibility webových služeb}

% 
% Texty abstraktů (anglicky, česky)
%
\abstracttexten{The text of the abstract (in English). It contains the English translation of the thesis title and a short description of the thesis.}

\abstracttextcz{Text abstraktu (česky). Obsahuje krátkou anotaci (cca 10 řádek) v češtině. Budete ji potřebovat i při vyplňování údajů o bakalářské práci ve STAGu. Český i anglický abstrakt by měly být na stejné stránce a měly by si obsahem co možná nejvíce odpovídat (samozřejmě není možný doslovný překlad!).
}

% Na titulní stranu a do textu prohlášení se automaticky vkládá 
% aktuální rok, resp. datum. Můžete je změnit:
%\titlepageyear{2016}
%\declarationdate{1. března 2016}

% Ve zvláštních případech je možné ovlivnit i ostatní texty:
%
%\university{Západočeská univerzita v Plzni}
%\faculty{Fakulta aplikovaných věd}
%\department{Katedra informatiky a výpočetní techniky}
%\subject{Bakalářská práce}
%\titlepagetown{Plzeň}
%\declarationtown{Plzni}

%%%%%%%%%%%%%%%%%%%%%%%%%%%%%%%%%%%%%%%%%%%%%%%%%%%%%%%%%%
%
% DODATEČNÉ BALÍČKY PRO SAZBU
% Jejich užívání či neužívání záleží na libovůli autora 
% práce
%
%%%%%%%%%%%%%%%%%%%%%%%%%%%%%%%%%%%%%%%%%%%%%%%%%%%%%%%%%%

% Zařadit literaturu do obsahu
\usepackage[nottoc,notlot,notlof]{tocbibind}

% Umožňuje vkládání obrázků
\usepackage[pdftex]{graphicx}
\graphicspath{{./img/}}

% Odkazy v PDF jsou aktivní; navíc se automaticky vkládá
% balíček 'url', který umožňuje např. dělení slov
% uvnitř URL
\usepackage[pdftex]{hyperref}
\hypersetup{colorlinks=true,
  unicode=true,
  linkcolor=black,
  citecolor=black,
  urlcolor=black,
  bookmarksopen=true}

% Při používání citačního stylu csplainnatkiv
% (odvozen z csplainnat, http://repo.or.cz/w/csplainnat.git)
% lze snadno modifikovat vzhled citací v textu
\usepackage[numbers,sort&compress]{natbib}


% na seznam zktratek
\usepackage{longtable}
\newcommand\nomenclature[2]{#1 & #2 \\}

% víceřádkové buňky v tabulce
\usepackage{makecell}

\renewcommand\theadalign{bc}
\renewcommand\theadfont{\bfseries}
\renewcommand\theadgape{\Gape[4pt]}
\renewcommand\cellgape{\Gape[4pt]}

%%%%%%%%%%%%%%%%%%%%%%%%%%%%%%%%%%%%%%%%%%%%%%%%%%%%%%%%%%
%
% VLASTNÍ TEXT PRÁCE
%
%%%%%%%%%%%%%%%%%%%%%%%%%%%%%%%%%%%%%%%%%%%%%%%%%%%%%%%%%%
\begin{document}
%
\maketitle
\tableofcontents

\chapter{Úvod}

- k čemu je práce dobrá
- co text práce obahuje
- use casy

\chapter{Principy webových služeb, techniky}
\label{sec:web-services-principles}

V této kapitole jsou definovány webové služby a související pojmy. Jsou zde představeny strojově čitelné způsoby popisu rozhraní služeb a protokoly sloužící ke komunikaci s webovými službami. Tato kapitola také obsahuje popis REST. 

%co je v téhle kapitole:
%
% - co jsou to webové služby
% - k čemu webové služby jsou
% - co je to REST (protože se to přímo dotýká mé práce)
% - co je to API
% - protokoly skrze které se s WS komunikuje
% - strojově čitelné formáty popisu ws

\section{Webové služby}

Pojem 'webová služba' má různé významy pro různé lidi, ale dá se najít několik společných bodů \cite{w3cWsDesignIssues}:
\begin{itemize}
	\item Použití HTML, XML a dalších standardů webové architektury jako stavebních kamenů
	\item Zaměření na podnikové a vnitropodnikové operace
\end{itemize}

% v citovaném článku je spousta zajímavých informací o WS, např "The fact that data is exchanged for business purposes and between different social entities means that accountability is required, rather than just reliable transmission. "
 - pro účely této práce je použita následující definice od W3C \cite{w3cWsArch}:
 "Webová služba je softwarový systém navržený pro podporu mezistrojové komunikace po síti. Webová služba má rozhraní, které je popsáno ve strojově čitelném formátu (konkrétně WSDL). Ostatní systémy interagují s webovými službami předepsaným způsobem za použití zpráv protokolu SOAP, které jsou typicky zprostředkované protokolem HTTP s využitím serializace XML a dalších webových standardů."
 
 % A Web service is a software system designed to support interoperable machine-to-machine interaction over a network. It has an interface described in a machine-processable format (specifically WSDL). Other systems interact with the Web service in a manner prescribed by its description using SOAP messages, typically conveyed using HTTP with an XML serialization in conjunction with other Web-related standards.

- poskytovatel služby = server

- konzument služby = klient (typicky nějaká aplikace)

% zmínit:
% zmínit RPC, SOAP
%- popis použitých technologií: XML, SOAP, WSDL

\section{REST}

%
% - co je REST
% - jak se liší od WS definované v předchozí sekci
% - základní popis elementů použitých v REST

- mohlo by se hodit: https://www.w3.org/TR/2004/NOTE-ws-arch-20040211/\#relwwwrest
- popisuje vztah REST a WWW
- REST: založeno na manipulaci s XML reprezentací webových resources skrze stateless operace

 - popis architektonického stylu REST: https://www.ics.uci.edu/~fielding/pubs/dissertation/rest\_arch\_style.htm
- popis elementů:
- data, konektory, komponenty
- popis view(na modelování)

- RFC na HTTP: https://tools.ietf.org/html/rfc7231\#section-4

- odkaz konkrétně na request methods, mohlo by se hodit, protože ty jsou indexovány, tak alespoň na citaci

- verzování API: https://www.xmatters.com/blog/devops/blog-four-rest-api-versioning-strategies/

- další verzování: https://www.troyhunt.com/your-api-versioning-is-wrong-which-is/

\subsection{Technologie použité k realizaci webových služeb}

- zmínit: RPC, SOAP

\begin{figure}
	\centering
	\includegraphics[width=\linewidth]{ws-tech-stack.png}
	\caption{Technologie zahrnuté ve webových službách}
	\label{fig:ws-tech-stack}	
\end{figure}

\section{Protokoly}

- relevantní protokoly: RPC, SOAP, HTTP

- HTTP (na REST a WS obecně)

- protokol aplikační vrstvy SOAP (web service)

- XML pro popis datového modelu

- specifikace SOAP: https://www.w3.org/TR/soap12-part1/\#intro

\section{Formální popis webových služeb}

- existují různé, strojově čitelné, formáty pro popis API

- WSDL 1.1, 2.0, WADL (REST), JSON-WSP

- Swagger, Raml, OpenApi

- v případě REST bohužel není nic formálně nutné (oproti třeba SOAP), takže specifikace API nemusí být kompatibilní, nemusí být úplné, nebo můžou být ad-hoc (např. slovní popis ve Word dokumentu) a tím pádem nemusí existovat univerzální způsob strojového čtení těchto specifikací

- v mé práci se věnuji především formátům WSDL, WADL a JSON-WSP
 
\chapter{Datové typy a porovnávání}

- přednášky z FJP
- jak jazyky řeší datové typy
	- rekurzivní vs. nerekurzivní
- primitivní typy (v xsd)
- built-in typy (v Jave)
- tady budu citovat \cite{abadi1995subytping}
	- subtyping: A <: B <=> A může být použito v kdekoliv kde je očekáváno B
	- kontravariance: F'(A) <: F(B) <=> B <: A

\subsection{Porovnávání datových typů}

 - jak to funguje
 - problémy při porovnání
 - subtyping vs. matching (\cite{abadi1995subytping})


\chapter{Získávání metadat webových služeb v CRCE}

Cílem této práce je vytvořit rozšíření pro úložiště CRCE\footnote{Component Repository supporting Compatibility Evaluation}, které bude schopno vyhodnocovat vzájemnou kompatibilitu indexovaných webových služeb. Tato kapitola představuje samotné úložiště, formát metadat a obecný způsob jejich získání. Na konci kapitoly je popsán princip fungování konkrétních rozšíření, která indexují webové služby (tzv. indexery).

\section{CRCE}

CRCE je komponentové úložiště dlouhodobě vyvíjené a spravované výzkumnou skupinou ReliSA na Katedře Informatiky ZČU, jehož primárním účelem je indexace a následná kontrola vzájemné kompatibility komponent. Úložiště je postaveno na modulární architektuře (viz obrázek \ref{fig:crce-arch}) a je tedy možné přidat rozšíření pro indexaci a zpracování vlastních dat.

\begin{figure}[h]
	\centering
	\includegraphics[width=10cm]{crce-arch.jpg}
	\caption{Architektura CRCE}
	\label{fig:crce-arch}
\end{figure} 

\subsection{Metadata komponent}
\label{subsec:crce-metadata}

Data, která vzniknou indexací komponenty a případným dalším zpracováním (např. porovnáním) jsou uložena do souboru metadat a představují klíčový element systému CRCE. Návrh struktury  těchto metadat, který je naznačen na obrázku \ref{fig:crce-resource-uml}, vychází z konceptu OBR\footnote{OSGi bundle repository} jehož základními entitami jsou mimo jiné \textit{Resource}, \textit{requirements} a \textit{capabilities}\cite{brada2015repository}. 

Entita \textit{Resource} reprezentuje komponentu uloženou v CRCE, \textit{requirements} a \textit{capabilities} jsou množiny vlastností, popisující co komponenta ke své správné funkci vyžaduje, respektive co naopak poskytuje. Model také umožňuje přidání key-value atributů k jednotlivým vlastnostem a jejich detailům.
 
 \begin{figure}[h]
 	\centering
 	\includegraphics{resource-uml}
 	\caption{Reprezentace metadat v CRCE}
 	\label{fig:crce-resource-uml}
 \end{figure}

V mé práci jsem pracoval především s poskytovanými vlastnostmi (množina \textit{capabilities}) a proto zde popíši hlavně jejich strukturu. Každá konkrétní vlastnost je reprezentována elementem \textit{Capability} a od ostatních je odlišena identifikátorem \textit{namespace}. Detaily konkrétní vlastnosti jsou popsány elementy \textit{Property} a \textit{Attribute}, kde \textit{Property} reprezentuje logický celek několika atributů. Například parametr endpointu REST služby shlukuje atributy popisující jeho jméno, datový typ atp. \textit{Attribute} pak představuje pár klíč-hodnota, který nese konkrétní informace jako např. jméno endpointu, nebo datový typ parametru.

Z obrázku \ref{fig:crce-resource-uml} je vidět rekurzivní povaha elementu \textit{Capability}, čehož je využito ke skládání jednodušších vlastností do složitějších celků. Vznikne tím stromová struktura, která je vhodná k modelování hierarchických dat mezi něž patří například popisy webových služeb. V případě takto komplexních vlastností je ke komponentně (\textit{Resource}) přiřazena pouze jedna, tzv. kořenová, \textit{Capability}, která reprezentuje celou vlastnost.

%Metadata v CRCE mají hierachickou strukturu:
%- Resource + Capability + Properties + Atributy
%- taky Requirements, ale ty v práci nepoužívám
%- Resource reprezentuje indexovanou komponentu
%- jednotlivé featury (indexovaná data) jsou reprezantovány stromem Capabilit
%- k Resource je vždy přiřazena root Capabilita
%- každá Capabilita má namespace, podle kterého lze určit co za konkrétní vlastnost popisuje
%- v případě root Capability by namespace měl být unikátní pro Resource (a resource by tedy neměl mít více root Capabilit s jedním namespace (snad?))
%- detaily fetatury jsou pak uloženy v dětských capabilitách, jejich Properties a Attributes
%- Capability mají Attributy + Properties
%- Properties mají atributy
%- Attributes pak nesou konkrétní hodnoty (Capability a Propeties slouží pouze jako jakési kontejnery)
%- Lze tak modelovat různé vlastnosti indexovaného objektu (viz \cite{brada2015repository}, tam je to dobře popsaný)
%- hierarchická struktura metadat je vhodná pro reprezentaci webových API, která jsou rovněž hierarchická

\subsection{Životní cyklus komponenty v CRCE}

Úložiště bylo původně navrženo pro ukládání OSGi komponent, nicméně indexovat lze jakoukoliv komponentu. Komponenta je v CRCE popsána již zmíněnými metadaty a prochází vlastním životním cyklem naznačeným na obrázku \ref{fig:crce-comp-lc}.

\begin{figure}[h]
	\centering
	\includegraphics{crce-component-lc.jpg}
	\caption{Životní cyklus komponenty v CRCE}
	\label{fig:crce-comp-lc}
\end{figure} 

Životní cyklus má dvě hlavní fáze, jimiž jsou \textit{Buffer} a \textit{Store}. Komponenta po nahrání do úložiště nejprve prochází fází \textit{Buffer}, v níž dojde k indexaci obsahu, kontrole vnitřní konzistence a kompatibility komponenty. Pokud touto fází projde bez chyb, je komponenta nahrána do trvalého úložiště (operace \textit{commit}) a přechází do fáze \textit{Store}.

V obou fázích jsou nad komponentou prováděny operace z nichž \textit{analyze} je nejvíce relevantní mé práci, protože právě během této operace dochází ke sběru metadat (fáze \textit{Buffer}) a dalším výpočtům nad nimi (fáze \textit{Store}). Modul s rozšířením, který je předmětem mé práce bude zařazen mezi výpočty prováděné nad metadaty během \textit{analyze}, kde bude vyhodnocovat vzájemnou kompatibilitu webových služeb. Způsoby sběru těchto metadat, ke kterým dochází ve fázi \textit{Store}, jsou popsány v následující sekci. 

%
%- primárně komponenta = OSGI bundle, ale může být cokoliv (např i pouhý textový soubor)
%
%- nahrání do bufferu
%
%- analýza
%
%- commit do store
%
%- další analýza
%
%- během toho se vytvářejí metadata
%
%- jednotlivé fáze mají callbacky, na které je možné v modulu navěsit vlastní funkcionalitu (\textit{BeforeUploadToBuffer}, \textit{AfterUploadToBuffer}, \textit{BeforeCommit}, ...)

\section{Indexování webových služeb}
\label{sec:api-index}

V této podkapitole je krátce popsán obecný způsob indexace komponent v CRCE. Následně je podrobněji rozebráno indexování webových služeb konkrétními moduly a reprezentace popisu API metadaty v CRCE. Na závěr jsou také uvedeny limity indexování.


\subsection{Obecná indexace komponenty}

Indexace komponenty a související sběr metadat je proveden ve fázi \textit{Buffer} k tomu určenými moduly. Ty jsou vzájemně nezávislé a obecně platí, že každý nich je zaměřen na sběr nějaké logicky ucelené části dat jako například informace o OSGi bundlu, maven koordináty, nebo popis webových služeb. Vlastní data komponenty zůstávají během tohoto procesu nezměněná což v kombinaci s řetězením indexerů zaručuje mimo snadnou rozšiřitelnost také transparentní přístup ke komponentě každému z nich. 

\subsection{Indexace komponenty s webovou službou}

Soubor obsahující implementaci, nebo popis API je v CRCE vnímán jako komponenta a prochází tedy zmíněným životním cyklem včetně výše popsané indexace. Pro popis webových služeb existuje mnoho standardních i nestandardních způsobů, jak již bylo zmíněno v kapitole \ref{sec:web-services-principles}. Z tohoto důvodu je není možné všechny analyzovat jedním indexerem a je nutné zaměřit se pouze na část z nich. 

V současné době tedy existují dva moduly podporující několik popisných formátů a implementací. Konkrétně se jedná o modul pro indexaci webových služeb založených na architektonickém stylu REST\cite{hessova2015rest} a o modul pro indexaci webových služeb s popisem ve formátu WSDL, WADL, nebo Json-WSP \cite{pejrimovsky2015ws}. Oba dva vznikly v rámci diplomových prací a jsou stručně popsány v následujících sekcích.

%- někde by asi bylo fajn ustanovit názvosloví použité v práci:
%\begin{itemize}
%	\item co je API: interface přístupné skrze síť (internet)
%	\item co je web service: service popsaný WSDL, WADL, nebo Json-WSP dokumentem
%	\item co je service: Service element in WSDL
%	\item co je endpoint
%	\item WSDL: port+operation
%	\item endpoint: REST, WADL, JSON-WSP
%\end{itemize}

\begin{figure}[h]
	\centering
	\includegraphics[height=11cm]{indexed-api-example}
	\caption{Příklad indexované SOAP webové služby pro komix Dilbert }
	\label{fig:indexed-api-example}
\end{figure}

\subsection{Struktura metadat popisující webovou službu}

Jak již bylo zmíněno v části \ref{subsec:crce-metadata}, hierarchickou strukturu popisu API lze vhodně vyjádřit metadaty CRCE. Během indexování komponenty reprezentující API jsou shromážděny různé typy popisných vlastností. Jedním z těchto typů je i samotný popis webové služby, který je reprezentován stromem metadat a ke komponentě je přiřazen skrze  kořenovou \textit{Capability}.   

I když jsou různé druhy API indexovány rozdílnými moduly, výsledná metadata mají podobnou strukturu. Příklad metadat API je zobrazen na objektovém diagramu \ref{fig:indexed-api-example}, jedná se o webovou službu, která vrací strip komixu Dilbert pro dnešní den.

Z uvedeného obrázku je vidět, že klíčové elementy API jako web service, nebo endpoint jsou reprezentovány objektem \textit{Capability}. Detaily těchto elementů jsou popsány objekty \textit{Property}. Jedná se zejména o parametry endpointů, těla requestů a response. Objekt \textit{Attribute} pak představuje konkrétní hodnoty, jež jsou na obrázku naznačeny jen jako páry "klíč=hodnota". \textit{Attribute} nemusí být vázaný jen na \textit{Property} a lze jej použít i pro popis \textit{Capability}, jak je tomu např. u objektu \textit{WebService1}.

\subsection{Indexování REST služeb}

Modul pro indexování REST API vznikl v rámci diplomové práce Bc. Gabriely Hessové. Princip sběru dat je založen na binární analýze java archivů (JAR) obsahujících implementaci REST služeb pomocí frameworků splňujících specifikaci JAX-RS a frameworku Spring Web MVC. Modul byl testován na frameworcích Jersey verze 2.26, RESTEasy verze 3.0.16 a  Spring Boot verze 1.5.9 \cite{hessova2015rest}.

Z implementace REST služby modul rekonstruuje kolekci endpointů s jejich parametry, tělem requestu, response a případnými parametry response. Každý endpoint je reprezentován entitou \textit{Capability}, všechny další jeho vlastnosti pak entitami \textit{Property}. Výčet všech indexovaných elementů rozhraní je uveden v tabulce \ref{tab:rest-indexed}.

\begin{table}[h]
	\centering
	\begin{tabular}{|l | c | c |}
		\hline
		Element API & Entita v metadatech & Vstaženo k\\
		\hline
		\hline
		Endpoint & Capability & - \\
		\hline
		Request body & Property & Endpoint \\
		\hline
		Request parameter & Property & Request \\
		\hline
		Response & Property Endpoint & Endpoint \\
		\hline
		Response parameter & Property & Response \\
		\hline
	\end{tabular}
	\caption{Seznam indexovaných elementů REST služby a jejich reprezentací v metadatech}
	\label{tab:rest-indexed}
\end{table}

\subsection{Indexování webových služeb na základě popisu}

Modul pro indexování webových služeb vznikl v rámci práce Bc. Davida Pejřimovského. Oproti předchozímu modulu pro indexaci REST služeb nepracuje tento s její implementací, ale s popisným souborem služby, tak jak bylo uvedeno v kapitole \ref{sec:web-services-principles}. Podporované formáty popisu služeb jsou WSDL (verze 1.1 i 2.0) pro SOAP webové služby, WADL a Json-WSP pro REST služby\cite{pejrimovsky2015ws}.

Struktura dat vytvořených pro REST služby (tedy z popisu WADL, nebo Json-WSP) je podobná struktuře dat vytvořenou předchozím indexerem. Endpoint je tedy reprezentován entitou \textit{Capability}, jeho parametry entitami \textit{Property}. Z popisu Json-WSP je ještě vytvořena reprezentace response pro daný endpoint (entita \textit{Property}). Z popisu WADL se žádné další vlastnosti endpointů nezískávají.

Z WSDL popisu je vytvořena reprezentace služeb, jež jsou popsány xml elementy \verb|<wsdl:service>| a jejich vnořených endpointů. Endpoint je ve WSDL popsán elementem \verb|<wsdl:port>| a má definované operace (elementy \verb|<wsdl:operation>|), nicméně modul tyto nevnořuje a vytváří zjednodušenou reprezentaci. Model endpointu tedy obsahuje metadata získaná z elementů \verb|<wsdl:operation>| a url definovanou v elementu \verb|<wsdl:port>|. Služby i endpointy jsou v metadatech reprezentovány entitami \textit{Capability}. Oproti REST službám, které mají jednu úroveň vnoření \textit{Capability}, zde vznikají úrovně dvě. Výčet elementů API a jejich reprezentace v metadatech je uveden v tabulce \ref{tab:ws-indexed}.

\begin{table}[h]
	\centering
	\begin{tabular}{|l | c | c |}
		\hline
		Element API & Entita v metadatech & Vstaženo k \\
		\hline
		\hline
		Service & Capability & - \\
		\hline
		Endpoint & Capability & Service \\
		\hline
		Endpoint parameter & Property & Endpoint \\
		\hline
		Response & Property & Endpoint \\
		\hline
		Response parameter & Property & Endpoint \\
		\hline
	\end{tabular}
	\caption{Seznam indexovaných elementů webové služby a jejich reprezentací v metadatech}
	\label{tab:ws-indexed}
\end{table}

Logika parsování WSDL souborů (verze 1.1. i 2.0) obsahovala chybu ve čtení adresy endpointu. Ta byla očekávána v atributu \verb|action| elementu \verb|<wsdl:operation>|, který ale není uveden ve specifikaci WSDL 1.1 \cite{wsdl1} ani WSDL 2.0 \cite{wsdl2}. Tuto chybu jsem v rámci mé práce opravil.

\subsection{Limity indexování}

Současný proces indexování webových služeb naráží na dva známé problémy týkající se datových typů. Jedná se o indexování rekurzivních datových typů a absenci samotných definic typů.

První problém se týká zejména indexování webových služeb podle popisných souborů, protože ty definice typů obsahují. Způsoby rozvoje a ukládání datových typů jsou popsány v \cite{abadi1995subytping}, nicméně logika zatím není implementována. Druhý problém se týká binární analýzy REST služeb, protože archiv s implementací služby nemusí nutně obsahovat definice tříd. Ty mohou být například v jiném artefaktu, na který se archiv pouze odkazuje skrze závislost.

Z těchto důvodů je do metadat uložen pouze název datového typu což snižuje možnosti porovnávání služeb.

% - custom datové typy
% - 2 problémy
%	- rekurzivní typy
%		- jsou způsoby pro jejich rozvoj: \cite{abadi1995subytping} a % ukládání
%		- nicméně indexovací logika není implementovaná (ani v jednom ze zmíněných indexerů)
%	- chybějící definice custom typů
%		- v případě např REST jsou uloženy v implementaci (nemusí se jednat ani o stejnou knihovnu) a indexer k nim nemusí mít přístup
%		- tím pádem je jméno datového typu (např. fully qualified name v případě Java třídy) jedinou informací, která je o typu dostupná
% - oba dva moduly používají jiné identifikátory pro stejné elementy (endpoint, parametr, ...) proto nelze vzájemně porovnat REST indexovaný prvním a druhým modulem

\chapter{Modul pro porovnávání webových služeb}

V této kapitole je detailně popsána funkce porovnávacího algoritmu společně s daty, nad kterými je možné porovnávač použít. Zároveň je zde popsán způsob vyhodnocení výsledků porovnání a formát uložení takto získaných dat.

\section{Reprezentace rozdílů a kompatibility}
\label{sec:diff-info}
Webové služby jsou popsány komplexní strukturou metadat, která byla uvedena v předchozí kapitole. Rozdíl mezi těmito strukturami lze vyjádřit pouhou pravdivostní hodnotou (stejné, nebo nestejné), nicméně takový přístup skrývá před klientem většinu informací, na základě kterých by se mohl rozhodnout o dalším postupu a tím značně snižuje použitelnost aplikace. Mimo to, samotná skutečnost, že metadata webových služeb se neshodují nemusí nutně z pohledu klienta znamenat nekompatibilitu. Z těchto důvodů je k reprezentaci rozdílů mezi službami třeba použít vhodnější metodu.

Článek \cite{brada2006diff} zabývající se možnostmi evaluace kompatibility komponent na základě relace subtypingu, nahlíží na komponentu skrze její rozhraní jako na datový typ a popisuje odlišnosti v různých úrovních kontraktu (metoda, parametr, ...). Na základě těchto odlišností (a relace subtypingu) je pak určena míra kompatibility dvou komponent. Protože je webová služba v CRCE reprezentována komponentou a na její rozhraní se dá také nahlížet jako na kontrakt datového typu, je tento přístup vhodný k reprezentaci výsledků porovnání webových služeb. Oproti pouhé pravdivostní hodnotě je navíc klientovi poskytnuta detailní informace o konkrétních rozdílech mezi službami. Na základě výše zmíněných důvodů jsem se rozhodl použít tento způsob reprezentace rozdílů v mé práci. 

Datové struktury použité pro reprezentaci rozdílů dvou entit a jejich vzájemné kompatibility navržené v rámci citovaného článku se nazývají \textit{Diff} a \textit{Compatiblity}. \textit{Diff} je definován jako rekurzivní typ, který uchovává jak konkrétní informace o rozdílu skrze podřazené \textit{Diff} tak i úroveň odlišnosti zvanou \textit{Difference}. Tyto úrovně jsou popsány tabulkou \ref{tab:diffs} a v citovaném článku tvoří obor hodnot funkce $diff(a,b): Type \times Type \rightarrow Difference$. Třída \textit{Compatibility} uchovává informace o kompatibilitě dvou komponent (reprezentovány entitami \textit{Resource}) společně s detaily jejich rozdílů, jež jsou reprezentovány stromem \textit{Diff}.

\begin{table}[h]
	\centering
	\begin{tabular}{|l|c|c|c|}
		\hline
		Název úrovně & Zkratka & Váha & Popis \\
		\hline
		\hline
		None & NON & 1 & $a = b$ \\
		\hline
		Insertion & INS & 2 & $a$ není definováno, ale $b$ ano \\
		\hline
		Deletion & DEL & 2 & $a$ je definováno, ale $b$ ne \\
		\hline
		Specialization & SPE & 3 & $b$ je subtyp $a$ ($b <: a$) \\
		\hline
		Generalization & GEN & 3& $a$ je subtyp $b$ ($a <: b$) \\
		\hline
		Mutation & MUT & 4 & kombinace \textit{INS}/\textit{SPE} a \textit{DEL}/\textit{GEN} \\
		\hline
		Unknown & UNK & 5 & $a$ nelze porovnat s $b$ \\
		\hline
	\end{tabular}
	\caption{Popis úrovní rozdílů}
	\label{tab:diffs}
\end{table} 

Způsob vyhodnocení rozdílů dvou webových služeb společně s významem jednotlivých úrovní \textit{Difference} a jejich vah pro klienta je popsán na konci této kapitoly. Pro přehlednost jsou jednotlivé úrovně rozdílů v průběhu kapitoly nazývané jejich zkratkami. 

%Popis výsledné datové struktury
%\begin{itemize}
%	\item Diff, Compatibility
%	\item vychází z \cite{brada2006diff}
%	\item stromová struktura rozdílů mezi jednotlivými uzly stromu metadat
%	\item obrázek \ref{fig:diff-construction} hezky popisuje jak to vznikne
%	\item výsledné hodnoty diffu a jejich významy pro klienta v tabulce \ref{tab:diff-level}
%	\item SPE/GEN může vzniknout jen z daových typů parametrů/response -> lze spolehlivě použít kontravarianci a výsledek obrátit
%	\item pokud tedy vyjde SPE, znamená to např generalizovaný parametr a tedy je to pro klienta bezpečné
%\end{itemize}

%\begin{figure}[h]
%	\centering
%	\includegraphics{diff-construction}
%	\caption{Vytvoření diffů}
%	\label{fig:diff-construction}
%\end{figure}

\section{Algoritmus porovnání}

Porovnávací algoritmus pracuje s metadaty popsanými v části \ref{subsec:crce-metadata}. Jedná se o stromovou strukturou, jejíž uzly tvoří instance tříd \textit{Capability}, \textit{Property} a \textit{Attribute}, kde objekty \textit{Attributes} jsou listy této struktury . Soubor metadat může obsahovat další vlastnosti komponenty, která představuje webovou službu, ta však zůstanou nedotčena, protože algoritmus pracuje pouze s daty, která byla vytvořena indexery popsanými v části \ref{sec:api-index}.

Moduly pro indexování webových služeb popsané v předchozí kapitole používají dvě různé množiny \textit{namespace} identifikátorů po pojmenování entit \textit{Capability}, \textit{Property} a \textit{Attribute}. Do budoucna je zároveň plánované rozšíření těchto modulů o funkcionalitu pro indexování datových typů a datová struktura metadat je tedy předmětem změny. Z těchto důvodů je algoritmus schopen porovnat pouze metadata vytvořená stejným indexerem a používající stejné \textit{namespace} identifikátory. Není tedy možné vzájemně porovnat například metadata REST služby získaná binární analýzou JAR s metadaty získanými čtením JSON-WSP dokumentu i když by se mohlo jednat o jednu službu.

\subsection{Popis algoritmu}

Vstupem algoritmu jsou reprezentace obou webových služeb v podobě objektů \textit{Resource}, z kterých jsou následně k porovnání vybrány kolekce endpointů, případě kolekce \textit{service} obsahující endpointy. Výstupem algoritmu je objekt \textit{Compatibility}, jež obsahuje detailní popis rozdílů mezi webovými službami ve formátu popsaném v sekci \ref{sec:diff-info}.

Před porovnáváním datových struktur je zkontrolována jejich kompatibilita. Ta je dána následujícími vlastnostmi:

\begin{itemize}
	\item typ popisu: z čeho byla metadata získána (implementace, formát popisného souboru),
	\item typ komunikace: jakým způsobem lze službu volat,
	\item \textit{identity capability}: obsahuje informace o identitě komponenty v CRCE.
\end{itemize}

Pokud kontrola proběhne úspěšně, je spuštěn samotný algoritmus jehož průběh je naznačen na vývojovém diagramu \ref{fig:apicomp-flow}. V případě porovnání SOAP webových služeb je tento postup obalen ještě porovnáním \textit{services}, které WSDL definuje a které jsou v CRCE reprezentovány jako kolekce endpointů.

\begin{figure}[h]
	\centering
	\includegraphics[width=9.5cm]{apicomp-flow}
	\caption{Vývojový diagram porovnávacího algoritmu}
	\label{fig:apicomp-flow}
\end{figure}

Mezivýsledky porovnání jsou ukládány v datové struktuře \textit{Diff}, která je detailně popsaná v sekci \ref{sec:diff-info}. Algoritmus postupuje po stromu metadat od kořene směrem k listům. V listech dojde k porovnání konkrétních hodnot. Rozdíl mezi dvěma uzly je vyhodnocen až po porovnání všech jejich potomků. Algoritmus tedy postupuje zpět ke kořeni stromu, který na konci algoritmu obsahuje finální údaj o rozdílu obou služeb. Jednotlivé fáze porovnání jsou popsány v následujících odstavcích.

\subsubsection{Výběr entit vhodných k porovnání}

Tento odstavec se týká endpointů a \textit{service}, protože tyto nejsou na pořadí závislé (na rozdíl např. od parametrů operace) a jejich pořadí v metadatech nelze ani předpokládat. Proto je nutné před samotným porovnáním nejprve vybrat dvojici entit k tomu vhodnou. Endpoint (\textit{service}) z první webové služby \textit{e1} (\textit{s1}) je vybrán sekvenčně (jak je naznačeno na \ref{fig:apicomp-flow}). Druhý endpoint \textit{e2} (\textit{s2}) je pak vybrán na základě určité shody s metadaty endpointu \textit{e1}, respektive \textit{service s2}. V přípdě endpointů se konkrétně jedná o počet povinných parametrů, jméno a URL na které je daný endpoint dostupný. URL nemusí být shodné úplně. Pokud tak nastane, je nastavena vlajka MOV, která je detailně popsána v sekci \ref{subsec:mov}.

O porovnatelnosti a rozdílu dvou \textit{services} je rozhodnuto na základě úplné shody jejich jmen a typu (současně je používán pouze typ "\verb|rpc/messaging|").

\subsubsection{Porovnání dvou endpointů}
Po výběru vhodné dvojice endpointů dojde k jejich porovnání. Postupně se porovnají (pokud jsou pro daný typ webové služby definovány) parametry, response a těla request. Způsob porovnání parametrů a response je rozveden v následujících odstavcích. Metadata těla request jsou dostupná pouze v případě REST služeb indexovaných na základě implementace a detail jejich porovnání je znázorněn v tabulce \ref{tab:req-body-cmp}. U atributu \textit{isOptional} může dojít ke změně \textit{GEN} pokud se povinný atribut stane nepovinným. V opačném případě se jedná o změnu \textit{SPE}.

\begin{table}[h]
	\centering
	\begin{tabular}{|l|c|}
		\hline
		Název atributu  & Možné výsledky \\
		\hline
		\hline
		\textit{isArray} & \textit{NON}, \textit{UNK} \\
		\hline
		\textit{isOptional} & \textit{NON}, \textit{GEN}, \textit{SPE} \\
		\hline
		\textit{dataType} & \textit{NON}, \textit{GEN}, \textit{SPE}, \textit{UNK} \\
		\hline
	\end{tabular}
	\caption{Porovnání atributů těl requestů endpointů}
	\label{tab:req-body-cmp}
\end{table}

\subsubsection{Porovnání response dvou endpointů}

Element response je definovaný ve všech případech kromě služeb popsaných WADL. V případě REST služeb indexovaných na základě implementace jsou navíc definovány i parametry response a může také existovat více elementů response pro jeden endpoint. Detail porovnání response (včetně parametrů) REST služeb je uveden v tabulce \ref{tab:resp-rest-cmp}, tabulka \ref{tab:resp-ws-cmp} pak obsahuje detail porovnání response ostatních služeb.

\begin{table}[h]
	\centering
	\begin{tabular}{|l|c|}
		\hline
		Název atributu & Možné výsledky \\
		\hline
		\hline
		\textit{isArray} & \textit{NON}, \textit{UNK} \\
		\hline
		\textit{dataType} &  \textit{NON}, \textit{GEN}, \textit{SPE}, \textit{UNK} \\
		\hline
		\textit{status} &  \textit{NON}, \textit{UNK} \\
		\hline
		\textit{parameterName} & \textit{NON}, \textit{UNK} \\
		\hline
		\textit{parameterCategory} & \textit{NON}, \textit{UNK} \\
		\hline
		\textit{parameterIsArray} &  \textit{NON}, \textit{UNK} \\
		\hline
		\textit{parameterDataType} & \textit{NON}, \textit{GEN}, \textit{SPE}, \textit{UNK} \\
		\hline
	\end{tabular}
	\caption{Porovnání atributů response a parametrů response endpointů REST služby}
	\label{tab:resp-rest-cmp}
\end{table}

\begin{table}[h]
	\centering
	\begin{tabular}{|l|c|}
		\hline
		Název atributu & Možné výsledky \\
		\hline
		\hline
		\textit{isArray} (pouze JSON-WSP) & \textit{NON}, \textit{UNK} \\
		\hline
		\textit{dataType} & \textit{NON}, \textit{GEN}, \textit{SPE}, \textit{UNK} \\
		\hline
	\end{tabular}
	\caption{Porovnání atributů response ostatních služeb}
	\label{tab:resp-ws-cmp}
\end{table}

\subsubsection{Porovnání parametrů endpointů}

Množiny atributů porovnávaných v případě parametrů endpointů jsou mezi službami navzájem různé, ale je tu průnik. Detaily v tabulce \ref{tab:param-cmp}.

\begin{table}[h]
	\centering
	\begin{tabular}{|l|c|c|}
		\hline
		Název atributu & Typ metadat služby & Možné výsledky \\
		\hline
		\hline
		\textit{name} & všechny & \textit{NON}, \textit{UNK} \\
		\hline 
		\textit{dataType} & všechny & \textit{NON}, \textit{GEN}, \textit{SPE}, \textit{UNK} \\
		\hline
		\textit{order} & všechny krom WADL & \textit{NON}, \textit{GEN}, \textit{SPE}, \textit{UNK} \\
		\hline
		\textit{isOptional} & všechny krom WSDL & \textit{NON}, \textit{GEN}, \textit{SPE}, \textit{UNK} \\
		\hline
		\textit{isArray} & REST a JSON-WSP & \textit{NON}, \textit{UNK} \\
		\hline
		\textit{category} & REST & \textit{NON}, \textit{UNK} \\
		\hline
	\end{tabular}
	\caption{Porovnání atributů response ostatních služeb}
	\label{tab:param-cmp}
\end{table}


\subsection{Porovnání datových typů}
V současnosti jsou největším omezením porovnávacího algoritmu datové typy. Jak již bylo zmíněno v kapitole \ref{sec:api-index}, jméno datového typu je jedinou dostupnou informací a tedy také jediným kritériem, podle kterého je lze porovnávat. To je dostatečné v případě vestavěných typů jako například třídy z balíku \verb|java.lang|, nebo typy definované v xsd. 

Nad těmito lze provádět plnohodnotné porovnání včetně kontroly generalizace (změny \textit{GEN}, \textit{SPE}). U vestavěných typů Javy je generalizace určena na základě dědičnosti a lze například určit, že datový typ s názvem \verb|java.lang.Number| je generalizací typu s názvem \verb|java.lang.Long|, protože třída \textit{Number} je rodičem třídy \textit{Long}. Obdobně je tomu i u vestavěných typů definovaných v xsd, kde je generalizace $a <: b$ definována v případě, že typ $b$ je dost velký na pojmutí typu $a$. Například typ \verb|xsd:long| dokáže reprezentovat číslo typu \verb|xsd:int| a proto platí, že $xsd:int <: xsd:long$. 

Porovnání vestavěných typů je založeno na podmínce správně zaindexovaného jména datového typu. V případě analýzy byte kódu je tato podmínka splněna, nicméně u metadat získaných čtením popisných XML souborů tomu tak vždy nemusí být. Problém spočívá v  předponě datového typu, která je součástí jeho jména a porovnávací algoritmus očekává její určitou hodnotu (konkrétně \verb|xs|). Hodnota předpony je však určena definicí namespace v XML popisu a ta se může od očekávané lišit. Tato definice není indexerem ukládána a není proto možné ji během porovnání přesně určit.

Pro uživatelsky definované typy je porovnání omezeno pouze na úplnou shodu, protože na základě pouhého jména datového typu nejde s jistotou usoudit nic dalšího. Změny pro jednotlivé rozdíly mezi datovými typy jsou uvedeny v tabulce \ref{tab:type-cmp}.

\begin{table}[h]
	\centering
	\begin{tabular}{|l|c|}
		\hline
		Vztah typů & Výsledek \\
		\hline
		\hline
		$ A = B$ & \textit{NON} \\
		\hline
		$ A <: B$ & \textit{GEN} \\
		\hline
		$ B <: A$ & \textit{SPE} \\
		\hline
		$ A != B$ & \textit{UNK} \\
		\hline
	\end{tabular}
	\caption{Rozdíly mezi datovými typy A a B}
	\label{tab:type-cmp}
\end{table}


\subsection{Složitost algoritmu}

Algoritmus v zásadě porovnává kolekce endpointů webových služeb a jejich počet je tedy hlavní parametr, od kterého se odvíjí výpočetní složitost. V případě SOAP webových služeb jsou operace definované v rámci \textit{service} a je tedy třeba brát v úvahu i jejich počet. Nárůst složitosti je ovlivněn výběrem vhodných párů entit k porovnání, konkrétní hodnoty pro nejlepší a nejhorší případy jsou uvedeny v tabulce \ref{tab:alg-complexity}.

\begin{table}[h]
	\centering
	\begin{tabular}{|l|c|}
		\hline
		Typ služby & Složitost \\
		\hline
		\hline
		REST, WADL, JSON-WSP & $\Omega(n)$, $O(n^2)$ kde $n$ je počet endpointů \\
		\hline
		WSDL & \makecell{$\Omega(mn)$, $O((mn)^2)$ kde $m$ je počet \textit{service}\\ a $n$ je počet endpointů} \\
		\hline
	\end{tabular}
	\caption{Složitost algoritmu}
	\label{tab:alg-complexity}
\end{table}
	
\section{Migrace webových služeb}	
\label{subsec:mov}
V popisu porovnání metadat endpointů bylo zmíněno použití URL endpointu jako identifikátoru. Tento přístup naráží na problém, pokud poskytovatel webovou službu přesune, nebo provede změny v cestě k danému endpointu. Na obrázku \ref{fig:mov-example} je uveden příklad dvou shodných endpointů, jež se liší pouze v URL. Aplikujeme-li výše popsaný algoritmus na tato data, skončí negativním výsledkem i přes to, že endpointy mají totožné rozhraní a klient by k nim mohl bez obtíží přistoupit. 

Podobným příkladem je i verze API v cestě k endpointu. Klient může mít požadavek na zjištění kompatibility API dvou různých verzí, ale algoritmus vrátí rozdíl \textit{MUT}, protože endpointy z prvního API vyhodnotí jako chybějící v API druhém a naopak, právě kvůli rozdílným cestám. Tím vznikne kombinace rozdílů \textit{DEL} a \textit{INS}, která vede na \textit{MUT}. Takové chování není žádané a je potřeba těmto problémům předcházet, proto je nutné případné změny v URL detekovat a brát je při porovnávání v potaz.

\begin{figure}[h]
	\centering
	\caption{Příklad shodných endpointů s rozdílnou URL}
	\label{fig:mov-example}
\end{figure}

Jako řešení popsaného problému byl zaveden příznak "MOV", který je ortogonální k úrovni změny (\textit{Difference}) popsané v předchozích sekcích a je součástí objektu \textit{Diff}. Díky ortogonalitě je možné stále určit méně nebezpečné rozdíly jako například vložený endpoint (INS), nebo generalizaci v příznaku parametru (SPE) a zároveň předat klientovi informaci o změně v URL. 

Příznak MOV má smysl brát v úvahu jen u případů porovnání jejichž úroveň změny je v podmnožině ${NON, GEN, SPE}$, která je v rámci této sekce označována jako bezpečná. Všechny ostatní úrovně představují v kontextu migrace příliš velkou změnu. Je tedy  například nesmyslné nastavit příznak MOV u rozdílu dvou endpointů s úrovní změny \textit{DEL}, protože samotná úroveň změny říká, že endpoint nebyl v druhé webové službě nalezen a proto nelze ani určit zda byl přemístěn.  

\subsection{Detekce změn v URL}

Součástí URL endpointu je také jeho název, změny v URL se tedy dají detekovat na třech místech. Prvním z nich je doména (včetně protokolu), druhým je cesta k endpointu a třetím jeho jméno. TODO co a jak se detekuje. 

\begin{figure}[h]
	\centering
	\caption{Třída nesoucí výsledek detekce změn v URL}
	\label{fig:mov-detect-res}
\end{figure}

Výsledek detekce změn v URL, jež je zachycen třídou \textit{MovDetectionResult} znázorněnou na obrázku \ref{fig:mov-detect-res}, nese tři příznaky, kde každý z nich určuje, zda byla v dané části URL detekována změna. Vzhledem k tomu, že algoritmus detekce pracuje pouze s URL a jmény endpointů, dojde k pozitivnímu výsledku i v případě, že jsou porovnávány dvě odlišné web služby. Aby se redukoval počet false-positives, je potřeba určit, které kombinace příznaků změny mohou vést na MOV a které představují příliš velké odchýlení. Tyto kombinace jsou uvedeny v tabulce \ref{tab:mov-explanation}, kde proměnné \textit{h}, \textit{p}, \textit{n} označují změnu v doméně, cestě a jménu endpointu.

\begin{table}[h]
	\begin{tabular}{|l|c|c|}
		\hline
		Kombinace & MOV & Zdůvodnění \\
		\hline
		\hline
		$!h \land !p \land !o$ & ne & Nebyla detekována žádná změna. \\
		\hline
		$h \land !p \land !o$ & ano & \makecell{Změna v doméně, může se \\jednat o migraci webové služby.} \\
		\hline
		$!h \land p \land !o$ & ano & \makecell{Změna v cestě k endpointům, může se \\ jednat o restrukturalizaci služby.} \\
		\hline
		$h \land p \land !o$ & ano & Může se jednat o kombinaci obou předchozích. \\
		\hline
		všechny ostatní & ne & Změna je příliš velká. \\
		\hline
	\end{tabular}
	\label{tab:mov-explanation}
\end{table}

Z tabulky je vidět, že změna ve jménech endpointů vždy vede na záporný výsledek. Důvodem je fakt, že jméno endpointu je druhé kritérium pro výběr vhodného páru endpointů (prvním je URL) a bez něj by algoritmus zdegradoval na porovnání "každý s každým" což by značně snížilo efektivitu.   

\subsection{Výběr entit k porovnání s příznakem MOV}

Protože jedním z kritérií pro výběr vhodného páru endpointů je i URL, je potřeba upravit logiku výběru tak, aby byly brány v potaz výsledky detekce popsané v předchozí sekci. Při porovnávání URL dvou endpointů je podle kombinace změn daná část URL ignorována a pracuje se pouze s nezměněnou částí u které je vyžadována striktní rovnost.

Tento způsob může vést na případy, kdy je dvojce endpointů vyhodnocena jako potencionálně vhodná k porovnání (s nastaveným příznakem MOV), nicméně porovnání skončí negativním výsledkem, například \textit{UNK}. Pouhá akceptace takového výsledku a pokračování algoritmu by mohlo zapříčinit negativní vyhodnocení kompatibility webových služeb v případech, kdy skutečný rozdíl není tak silný. TODO: příklad 

\subsubsection{Algoritmus 'pick best'}
Řešením zmíněného problému je algoritmus 'pick  best', který postupně porovnává dvojice endpointů (k tomu předem vybrané) a jako výsledek vrátí dvojici s nejlepším rozdílem. Vstupem algoritmu je tedy endpoint z první webové služby \textit{e1} a množina endpointů druhé webové služby \textit{endpoints2}. 

Algoritmus postupně vybírá elementy \textit{e2} z \textit{endpoints2} a pokud je pár $e1,e2$ vyhodnocen jako porovnatelný, dojde k detailnímu porovnání. V případě výsledku spadajícího do bezpečné množiny (${NON, GEN, SPE}$) je pár $e1,e2$ vrácen a porovnání pro endpoint \textit{e1} je ukončeno. V opačném případě je negativní výsledek uložen a z množiny \textit{endpoints2} jsou vybírány další porovnatelné endpointy dokud algoritmus nedojde ke dvojici, jejíž rozdíl spadá do bezpečné množiny, nebo dokud není \textit{endpoints2} vyčerpána. Pokud je množina \textit{endpoint2} vyprázdněna, znamená to (alespoň částečnou) nekompatibilitu obou webových služeb a je vrácen první porovnávaný pár $e1,e2$.

\subsection{Verze REST API v cestě k endpointu}	
\label{sec:api-path-version}

Jednou ze speciálních změn detekovatelných v cestě k endpointu je verze API. Jedná se o běžnou praktiku \cite{restApiVersion} a k jejímu zpracování lze použít jednodušší přístup, než byl dosud popsán. Část cesty, která obsahuje verzi lze jednoduše vypustit a porovnat URL bez verze, což případě stejného API (s odlišností verze) znamená porovnání dvou identických URL. Příznak MOV je potom nastaven pokud jsou URL s verzemi rozdílné a URL bez verzí stejné.

Algoritmus podporuje standardní verzovací formát \verb|major.minor.micro|, který je vyjádřen regulárním výrazem: \verb|/[vV][0-9]+(?:[.-][0-9]+){0,2}/|.

\section{Vyhodnocení výsledků}

Mezivýsledky porovnání jednotlivých elementů stromu metadat jsou ukládány do hierarchické struktury \textit{Diff} a vždy po vyhodnocení všech rozdílů potomků dvou uzlů, dojde na základě těchto i k vyhodnocení rozdílů uzlů samotných. Úroveň rozdílu jež určuje finální kompatibilitu webových služeb je dána úrovní kořenového \textit{Diff}, který drží celou strukturu popisující detailní rozdíly mezi službami. 

Příklad tvorby \textit{Diff} při porovnání dvou operací je znázorněn na obrázku \ref{fig:compatibility-creation}. Nejdříve dojde k vyhodnocení rozdílů listů, tedy atributů \textit{name} a \textit{type} parametrů obou operací, čímž vzniknou \textit{type diff} a \textit{name diff}. Na základě těchto se určí rozdíl mezi parametry samotnými (\textit{parameter diff}) a po porovnání metadat operací (\textit{metadata diff}) je vyhodnocen i výsledný rozdíl obou operací jež je reprezentován objektem \textit{endpoint diff}.

\begin{figure}[h]
	\centering
	\includegraphics[height=6.5cm]{compatibility-construction}
	\caption{Tvorba rozdílů mezi dvěma endpointy}
	\label{fig:compatibility-creation}
\end{figure}

Vyhodnocení úrovně rozdílu \textit{Diff} na základě jeho potomků je řízeno prioritou. Každá z úrovní rozdílů \textit{Difference} popsaných v tabulce \ref{tab:diffs} má určitou prioritu, která reprezentuje závažnost rozdílu a uzel od svých potomků vždy přejímá \textit{Difference} s nejvyšší prioritou. Tím je zaručeno, že se rozdíly narušující kompatibilitu projeví na finálním verdiktu.

Pokud tedy například dojde k rozdílu \textit{UNK} (maximální priorita) při porovnání dvou atributů parametru endpointu, postupným vyhodnocováním se tato \textit{Difference} dostane až ke kořenovému uzlu a výsledná kompatibilita bude mít hodnotu \textit{UNK}. Hodnoty jsou v tabulce \ref{tab:diffs} seřazeny od nejnižší priority po nejvyšší.


\subsection{Dopad na klienta}

Výše popsané úrovně \textit{Difference} mají rozdílný dopad na klienta. Ten se dá rozdělit do skupin bezpečné, potenciálně nebezpečné a nebezpečné, tak jak je tomu v tabulce \ref{tab:diff-level}. Následující odstavce popisují dopad na klienta pro jednotlivé úrovně a zdůvodňují jejich zařazení do dané skupiny.

\begin{table}[h!]
	\centering
	\begin{tabular}{|l|c|}
		\hline
		Difference & Dopad na klienta  \\
		\hline
		\hline
		None (NON) & bezpečné \\
		\hline
		Specialization (SPE) & bezpečné  \\
		\hline
		Insertion (INS) & bezpečné \\
		\hline
		Deletion (DEL) & potenciálně nebezpečné \\
		\hline
		Generalization (GEN) & potenciálně nebezpečné \\
		\hline
		Mutation (MUT) & nebezpečné \\
		\hline
		Unkown (UNK) & nebezpečné \\
		\hline
	\end{tabular}
	\caption{Dopad jednotlivých úrovní rozdílu na klienta }
	\label{tab:diff-level}
\end{table}

\subsubsection{Bezpečné rozdíly}

Pokud je rozdíl mezi službami v bezpečné skupině, může klient transparentně volat obě služby, aniž by došlo k jakékoliv chybě v rámci kontraktu. Úroveň \textit{SPE} sice označuje specializaci, nicméně změny \textit{SPE} a \textit{GEN} mohou nastat pouze v případě neshody datových typů parametrů, nebo odpovědi operace. V takovém případě je aplikován princip kontravariance \cite{abadi1995subytping} podle kterého platí, že $F'(a') <: F(a) <=> a <: a'$ a \textit{SPE} tedy znamená bezpečnou změnu \textit{GEN} u parametrů, nebo odpovědi operace. 

Změna \textit{INS} nastává v případě, že druhá webová služba obsahuje operace, které nejsou v první definované. Vzhledem k tomu, že klient nemohl volat a nemůže tak dojít k porušení kontraktu webové služby, je tato změna brána jako bezpečná.

\subsubsection{Potenciálně nebezpečné rozdíly}

Změny v této skupině mohou mít nebezpečný dopad na klienta ve smyslu volání operace webové služby s neplatným kontraktem, nicméně nejsou tak závažné jako změny nebezpečné. Posouzení reálného rizika změny provádí klient na základě vráceného objektu popisujícího kompatibilitu služeb.

Úroveň \textit{DEL} označuje element definovaný v první službě, ale chybějící v druhé. Může se jednat například o parametr operace, nebo operaci samotnou. Úroveň \textit{GEN} je získána, stejně jako \textit{SPE}, na základě principu kontravariance a označuje změnu v datovém typu (parametru, odpovědi operace, ...), například z \textit{long} na \textit{int}. 

\subsubsection{Nebezpečné rozdíly}

Změny v této skupině představují buď kombinace několika méně závažných změn (\textit{MUT}), nebo neporovnatelnost obou webových služeb (\textit{UNK}). Stejně jako u předchozí skupiny, i zde může klient použít výsledky porovnání a sám rozhodnout o míře nekompatibility. Úroveň \textit{MUT} může například nastat i kombinací rozdílů endpointů, které klient nepoužívá a druhá webová služba tak pro něj může být stále kompatibilní.

Úroveň \textit{UNK} označuje neporovnatelnost metadat a v takovém případě nelze o kompatibilitě jednoznačně rozhodnout.


\chapter{Implementační detaily (jen stručně)}

 - zmínit, proč třídy pro porovnávání REST API a WS nemají společného předka (krom rozhraní)
 	- důvod: chtěl jsem nechat implementaci obou porovnávačů oddělenou pro případ, že by se změnila funkce indexerů

\chapter{Testování}

- nějaká reálná data
	- STAG (WSDL)
	- Fuel Economy
- i syntetická data
- algoritmus testován pomocí unit testů

\section{Integrační + akceptační testy}

- testování skrze REST API
- pomocí Postman (Collection Runner)
- několik verzí jednoho API -> testování křížem
- todo: příklad
	- popsat verze API (čím se liší), případně zdůvodnit očekávaný výsledek
	- tabulka vzájemného porovnání s výsledky

\subsection{Syntetický server v Jave}
\begin{itemize}
	\item postaveno na Jersey
	\item REST API
	\item alespoň obrázek/raml api?
	\item 2 verze otestované křížem
\end{itemize}

Popis verzí API (rozdíl je vždy popsaný oproti verzi 1):
\begin{itemize}
	\item V1: základní verze
	\item V2: typ parametru změněn z Long na Number
\end{itemize}

Co se testuje:
\begin{itemize}
	\item GEN/SPEC a kontravariance u parametru endpointu
\end{itemize}	
	
\subsection{Příklad JSON-WSP z wiki}

\begin{itemize}
	\item popsáno JSON-WSP souborem
	\item https://en.wikipedia.org/wiki/JSON-WSP
	\item 4 verze otestované křížem
\end{itemize}

Popis verzí API (rozdíl je vždy popsaný oproti verzi 1):
\begin{itemize}
	\item V1: základní verze
	\item V2: typ User je jiný
	\item V3: přidána metoda deleteUser
	\item v4: metoda listGroups změněna na getUsersInGroup
\end{itemize}

Co se testuje:
\begin{itemize}
	\item ne-indexování custom datových typů (takže v2 = v1)
	\item mutace endpointu
	\item INS, DEL
\end{itemize}

\subsection{FuelEconomy API}

\begin{itemize}
 \item popsáno WADL souborem
 \item https://www.fueleconomy.gov/ws/rest/application.wadl
 \item 3 verze otestované křížem
\end{itemize}

Popis verzí API (rozdíl je vždy popsaný oproti verzi 1):
\begin{itemize}
	\item V1: základní verze
	\item V2: odebrán (poslední) resource /labelvehicle
	\item V3: odebrán (poslední) resource /labelvehicle, přidán resource /somethingDifferent
\end{itemize}

Co se testuje:
\begin{itemize}
	\item  INS, DEL a MUT
\end{itemize}

\subsection{Stag WS}

TODO

\chapter{Závěr}	

 
% 
% PRO ANGLICKOU SAZBU JE NUTNÉ ZMĚNIT
% CITAČNÍ STYL!
%
\bibliographystyle{csplainnatkiv}
{\raggedright\small
\bibliography{literatura}
}

% seznam zkratek
\chapter*{Seznam zkratek}
\addcontentsline{toc}{chapter}{Seznam zkratek}

\begin{longtable}{@{}p{3cm}@{}p{\dimexpr\textwidth-1cm\relax}@{}}
	\nomenclature{CRCE}{Component Repository supporting Compatibility Evaluation}
	\nomenclature{API}{Application Programming Interface}
	\nomenclature{HTTP}{Hypertext Transfer Protocol}
	\nomenclature{JAR}{Java Archive}
	\nomenclature{JSON}{JavaScript Object Notation}
	\nomenclature{JSON}{JSON Web-Service Protocol}
	\nomenclature{OSGi}{Open Services Gateway initiative}
	\nomenclature{REST}{Representational State Transfer}
	\nomenclature{SOAP}{Simple Object Access Protocol}
	\nomenclature{UML}{Unified Modeling Language}
	\nomenclature{URI}{Uniform Resource Identifier}
	\nomenclature{URL}{Uniform Resource Locator}
	\nomenclature{WADL}{Web Application Description Language}
	\nomenclature{WSDL}{Web Services Description Language}
	\nomenclature{XML}{Extensible Markup Language}
	\nomenclature{XSD}{XML Schema Definition}
\end{longtable}

\chapter*{Příloha A - WSDL webové služby pro komix Dilbert}
\addcontentsline{toc}{chapter}{Příloha A - WSDL webové služby pro komix Dilbert}
Tady bude WSDL

\end{document}
